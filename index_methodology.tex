\documentclass{article}
\usepackage[utf8]{inputenc}
\usepackage[T1]{fontenc}
\usepackage{lmodern}
\usepackage{hyperref} %url
\usepackage[numbers]{natbib}
\usepackage{amsmath}
\usepackage{enumitem}
\usepackage{mathtools}
\usepackage{booktabs} % nice tables
\usepackage{float} % position tables [H]
\usepackage{authblk}
\usepackage{array}
\newcolumntype{L}{>{\centering\arraybackslash}m{3cm}}


\DeclarePairedDelimiter{\abs}{\lvert}{\rvert}

\title{\textbf{BlockchainX indices}}

\author{\textbf{Vinter Capital} \\ Version 0.2}



\begin{document}



\maketitle

\pagebreak

\tableofcontents


\section{About Vinter Capital}

We believe that blockchain technology has the potential to transform our world. The application of this technology can result in a more decentralized and transparent economy with far-reaching positive effects such as financial inclusion for everyone.

The objective of Vinter Capital is to 
lower the entry barriers that exist today when investing in cryptocurrencies. 
We do so by providing transparent and regulated indices that can be used in financial products.
Our products have the potential of being a bridge between the professional investment industry and the cryptocurrency community.

% ev infoga ngt om Cryptocurrency is a new asset class ... kommer inte på något nu. 

\section{BlockchainX indices}

The cryptocurrency market is in its infancy. This presents a challenge to established methods for indexing. BlockchainX indices combine the best of traditional index methodologies with appropriate adjustments for cryptocurrency factors such as: liquidity, capital controls, exchange stability and custody limitations. 

The BlockchainX index family is developed to provide a rule-based and transparent way to track the value of cryptocurrencies. All indices are designed to be regulated investable benchmarks.

All BlockchainX indices are compliant by design. 
Decisions must be clear, rule-based, robust, reliable and transparent.
  The methodology is developed, operated and administered transparently in compliance with Article 13 of the Regulation 2016/1011 on indices used as benchmarks in financial instruments and financial contracts or to measure the performance of investment funds (the ”BMR”). The key elements of this methodology are published and made available for each benchmark provided and published or, when applicable, for each family of benchmarks provided and published.

% har ej deft index committee så stryker detta: if the judgment of the BlockchainX Index Committee is required when applying these rules, decisions will be made public with full documentation of the decision-making process.

This methodology states the regulatory framework for the development, calculation and administration of the BlockchainX index family.

\section{Input data}
\subsection{Input data requirements}
Vinter Capital’s provision of benchmarks shall be governed by the following requirements in respect to its input data:
\newline
\begin{itemize}
    \item the input data shall be sufficient to represent accurately and reliably the market or economic reality that the benchmark is intended to measure;
    \item The input data shall be transaction data, if available and appropriate. If transaction data is not sufficient or is not appropriate to represent accurately and reliably the market or economic reality that the benchmark is intended to measure, input data which is not transaction data may be used, including estimated prices, quotes and committed quotes, or other values;
    \item the input data shall be verifiable; 
    \item clear guidelines regarding the types of input data, the priority of use of the different types of input data and the exercise of expert judgment shall be published;
    \item where a benchmark is based on input data from Contributors, Vinter Capital will obtain, where appropriate, the input data from a reliable and representative panel or sample of Contributors so as to ensure that the resulting index is sufficient to represent accurately and reliably the market or economic reality that the benchmark is intended to measure. 
\end{itemize}
Vinter Capital will not use input data from a contributor if the administrator has any indication that the contributor does not adhere to the code of conduct referred to in Article 15 of the BMR, and in such a case shall obtain publicly available  representative data.

\subsubsection{Contributor selection}
The quality of data contributors is assured through the following controls: 
\begin{itemize}
    \item Presumable Contributors are evaluated on the basis of data quality, cost of sources, reputation and market share.
    \item Input data is compared between multiple Contributors in order to ensure its integrity and accuracy. In the event of data being insufficient or unverifiable, one Contributor will be replaced with another. 
\end{itemize}

\subsection{Continuous evaluation of selected contributors}
Reliability of provided data is assessed with respect to availability and consistency of each data source. Data is compared across multiple independent data providers. Anomalies, such as abnormal deviation from average, are investigated. Providers with substantial and persistent anomalies are at risk of being removed as data contributors to the BlockchainX indices. Accuracy is verified by comparing contributed data with other trusted data sources, such as ECB, to which the associations patterns are considered to be known. Furthermore, computation schemes such as free-float schemes are also compared, both from a qualitative as well as quantitative perspective, between independent scheme providers.

\subsection{Data correction procedure}
In the case of data corruption, Vinter Capital will immediately inform stakeholders concerning the error. An investigation into the reasons behind the corrupted data will take place in order to remove possible vulnerabilities from data-collection processes. Erroneous computations are corrected whenever possible. Furthermore, a consequence analysis will be performed where financial and legal consequences, with respect to corrupted data, are assessed and a structural review of relevant computational schemes are performed. Affected clients will then be informed about the error, its potential legal and financial consequences and relevant recalculations. Any conduct that may involve manipulation or attempted manipulation of an index is reported to regulators.

\subsection{Circulating supply}
Circulating supply for each constituent is determined via:
\begin{enumerate}
\item The summation of all Unspent Transaction Outputs (UTXOs). The summation is based on the transaction history of a full node controlled by the BlockchainX index committee.
\item Iterating over all mined blocks and summing all newly minted cryptocurrency. The summation is based on the transaction history of a full node controlled by the BlockchainX index committee. 
 \item Public Blockchain explorers selected and reviewed by the BlockchainX index committee.
 \item Public code and documentation reviewed by the BlockchainX index committee.
 \item Other selected data contributors. 
 \end{enumerate}


\subsection{Free-float}
Free-float equals circulating supply with a potential reduction due to one or more of the following factors:

\begin{itemize}
 \item The amount of cryptocurrency allocated prior to the public release
   of a Blockchain and that remain in the control of developers, principals,
   foundations or corporations.
\item Cryptocurrencies that are deemed not accessible to any market
  participant due to loss of private keys, dust-accounts (accounts with lower holdings than the current fees or cost associated with transfers or creation of a wallet), or burning (a strategy that
  seeks to obtain a price increase by directly reducing the circulating supply of
  a cryptocurrency). 
\item Other factors as determined by the BlockchainX index committee.
\end{itemize}
Issuance will be returned to the free-float circulation in the event of a public announcement that assets have been sold into the public market. Free-float adjustments are made on the monthly rebalancing date.


\subsection{Foreign exchange rates}
Foreign exchange reference rates are obtained from ECB daily at 16:00 CET. \footnote{\url{www.ecb.europa.eu/stats/policy_and_exchange_rates/euro_reference_exchange_rates}}

\subsection{Selected exchanges}\label{sec:selected-exchanges}
Selected exchanges contribute market data to the computation of the BlockchainX indices. As of \today, market data is obtained from the following cryptocurrency exchanges: 
Bitfinex, OKCoin, Bitstamp, Itbit, Coinbase, Coinbase Pro, Kraken and Gemini.
For an exchange to be selected as a data contributor it must have:

\begin{enumerate}
% trading requirements -------------------
\item been operating as a cryptocurrency exchange for a minimum of two years.
\item for at least one month, implemented trading, deposits and withdrawal fees
\item met a minimum monthly liquidity threshold with respect to total trading volume.
\item for at least one month, provided reliable and valid market data.

%\item quotes in USD.
\item for at least one month, offered the possibility to withdraw and deposit USD.
% legal requirements --------------------
\item chosen a jurisdiction of incorporation that offers sufficient investor protection.
\item fulfilled all applicable regulatory frameworks such as know your customer and anti-money laundering requirements.
\item provided information concerning ownership and corporate structure. 
\item passed the BlockchainX index committee's risk, security and suitability review. The review includes an evaluation of past security breaches, trading cessations, legal disputes and if provided market data are to be considered readily available. 
\end{enumerate}

Under extraordinary circumstances, exchanges can be added or removed as data contributor at the discretion of the BlockchainX index committee. 

\section{Index methodologies}

The BlockchainX index family consists of several indices. 
The BL10M-U index contains the 10 largest assets, weighted by market capitalization, and is denominated in USD. 
The BL5E-S index contains the 5 largest assets. It is equally weighted and denominated in SEK. 
All indices are listed in the appendix.

Every index is priced using a 20-second average of the BlockchainX composite constituent price (described in \ref{sec:BCP}). Pricing occurs with 20-second intervals between 00:00 and 24:00 CET.

A daily closing value is calculated at 17:00 CET. 
The index value is published in USD, EUR and SEK using exchange rates from ECB. 

% todo ev infoga en summary över tratten vi har. som är 1 välj exchanges 2 lista alla coins 3 filtrera bort dom som ej möter eligibilty criterias 4 sortera på marketcap 5 vikta.

\subsection{Eligible constituents} \label{sec:eligible-const}
Cryptocurrencies trading on \textit{selected exchanges} are eligible as index constituents in BlockchainX indices if they:

\begin{enumerate}
\item allow for air-gapped cold storage, including offline wallet generation and offline transaction signing.
\item have not been pegged to another asset such as currencies or commodities.
\item have for at least one month been traded on two selected exchanges.
\item can be deposited and withdrawn from at least two selected exchanges.
\item are not an ongoing Initial Coin Offering (ICO).
\item have at least 20\% of its monthly trading volume located at selected exchanges.
\item have no more than 90\% of its monthly trading volume located at a single selected exchange.
\item have a monthly trading volume that exceeds 20\% of its circulating supply. 
\item have at least a daily volume of USD 20 million over the past month. 
\item have not been deemed a security, or potential security, by the BlockchainX index committee.
\item have not been deemed fraudulent by the BlockchainX index committee.
\item are a cryptographically secured digital bearer instrument.
\item are freely traded and can be freely held for the foreseeable future.
\item maintain an underlying protocol that has been deemed technically and cryptographically sound with no known security vulnerabilities, including critical bugs, undue exposure to 51\% attacks, or other factors as determined by the BlockchainX index committee.
\end{enumerate}

Cryptocurrencies that meet these criteria are \textit{eligible constituents}. 


% If there is not enough eligible constituents meeting these eligibility criterias, the policy of 3.4 Index constituent selection will instead utilized. denna var otydlig så jag tog bort den 


\subsection{Selected constituents}

\textit{Eligible constituents} (section \ref{sec:eligible-const}) are ranked by market capitalization in descending order. 
For indices with 10 constituents, the top 8 of the ranked list are selected immediately.
Secondly, constituents that (a) were selected at the previous rebalancing date and (b) have a rank between 9 and 12 are added to the selected constituents. Remaining constituents are chosen according to their market capitalization. 

For indices with 5 constituents, the top 3 of the ranked list are selected immediately.
Secondly, constituents that (a) were selected at the previous rebalancing date and (b) have a rank between 4 and 7 are added to the selected constituents. Remaining constituents are chosen according to their market capitalization.

If it is not possible to reach the intended number of constituents, the BlockchainX index committee can decide to either include non-eligible constituents or allow the index to have less constituents than intended.

\subsection{Constituent price (BCP)}\label{sec:BCP}
All eligible %eller bara selected?
cryptocurrencies are priced using the BlockchainX composite price algorithm. In order to compute the BlockchainX composite constituent price (BCP), the algorithm executes three steps.

First, at time $t$, order data on executed trades are obtained from all selected exchanges with respect to a 20-second time window.

Secondly, for each exchange and constituent, a Volume Weighted Average Price (VWAP) is computed with respect to executed trades within the specified time window. 

Thirdly, the median of all exchange specific VWAP:s is taken as the BCP for each constituent. 

Missing data is imputed through a nearest neighbor approach with respect to time. Imputation is performed using data from all selected exchanges. All BCPs are computed using USD as quote currency. The BCP is then translated into SEK and EUR using foreign exchange reference rates from ECB.

\subsection{Index price and weights}

% här kommer index ekvationen. 

% det som är lurigt är att mcap och ew weighted får olika ekvationer. vi kan lösa det med en subsub section eller bara säga i stycket "for an equal weighted index the weight in each asset is $w_i = 1/k$.

Let the number of selected constituents at time $t$ be denoted as $k(t)$. 
Let $w_i(t)$ and $p_i(t)$ be the weight and BCP of asset $i$.
The BlockchainX index price is then given by:
\begin{equation}
  % \text{BLX}(t) =  
  \frac{  \sum_{1}^{k(t)} w_i(t) p_{i}(t)  }{\text{DIV}(t)}
\label{eq:index}
\end{equation}
where $\text{DIV}(t)$ is a divisor (described in section \ref{sec:divisor}). The weight of asset $i$ at time $t$ is:
\begin{equation}
w_i(t)=
\begin{cases}
    1 / k(t), & \text{for equally weighted indices} \\
    f_{i}(t) /f(t),  & \text{for market capitalization weighted indices}
\end{cases}
\end{equation}
where  $f_{i}(t)$ is the free-float of constituent $i$ and $f(t)=\sum_1^{k(t)}f_i(t)$, both are recalculated at rebalancing date.
%For an equal weighted index  $w_i(t) = 1 / k(t)$. 
%For a market capitalization weighted index $w_i(t) = c_{i}(t) f_{i}(t)$ where  $c_{i}(t)$ is the circulating supply and $f_{i}(t)$ is a free-float factor.

\subsection{Divisor}\label{sec:divisor}

Index adjustments, such as monthly rebalancing, should not change the index value. A divisor is therefore introduced in order to insure that the index value only fluctuates due to price movements in the underlying assets and not due to other events that affect total market capitalization.
% The divisor is a function of all adjustment factors, prices, circulating supplies as well as time.

At inception, the divisor is given by
\begin{equation}  
  \text{DIV}(0) = \frac{1}{K} \sum_{i}^{k(0)} w_i(0) p_{i}(0)
  \label{eq:initial_divisor}
\end{equation}
where $K=100$ in order to ensure an initial index value of 100. 

A fee of 2.5 percent per annum is deducted from the divisor on a daily basis. The closing index price is published daily at 17:00 CET, and the fee is accounted for by multiplying the previous day's divisor with $(1 + 0.025/365)$.

Given a positive number $\delta$, the divisor can be calculated for any time $t$ recursively via
\begin{equation}
  \text{DIV}(t) 
  = \frac{\sum_{i}^{k(t)}
    w_{i}(t)p_{i}(t)}{\sum_{i}^{k(t-\delta)}
    w_{i}(t-\delta)p_{i}(t-\delta)} 
    \text{DIV}(t-\delta)
\label{eq:divisor}
\end{equation}
which ensures index continuity. 
\subsection{Rebalancing}

BlockchainX indices are rebalanced monthly. All weights $w_i(t)$ have identical numerical values between rebalances. 

The rebalancing window is set to 12:00 CET on the first business day of the month, plus or minus 24 hours. 
Rebalancing will occur at a randomly chosen time during this window. 
Randomization is used in order to avoid front-running. 
Clients subscribed to any of Vinter Capital's indices will receive an email containing information about the new weights for all assets in that index.
On the second business day of the month at 16:30 CET, the actual weights are published on Vinter Capital's website. The delay is implemented in order to increase the investability of BlockchainX indices.

Rebalancing involves 
(i) a review of exchanges,
(ii) selecting constituents  
and (iii) calculating their weights.

% The procedure for updating w_i(t) starts with a review of exchanges in accordance to the criterias outline in section qq ref selected exchanges. All cryptocurrencies listed on \textit{selected exchanges} that meet the eligibility criterias (enumerated in section qq ref eligible constituents) are deemed to be  \textit{eligible constituents} and this subset of cryptocurrencies are ranked according to market capitalization.  

\subsection{Market events}

% this should maybe be in appendix. the reader has not yet come to the actual formula for weights. think about the reader journey. 

Cryptocurrencies have a series of unique market events, compared to traditional assets, such as forks, staking and airdrops. These events have the potential to disrupt as well as increase the value and security of current selected constituents. However, in order to reduce unpredictable changes in the composition of the index, the BlockchainX index committee will handle each event with the intention that intra monthly index return should solely depend on price movements in index constituents. 

\subsubsection{Forks}

Formally, a Blockchain is a network of computers that have installed the same software in order to manage a distributed database of transaction history. To own cryptocurrency is equivalent to having writing permission to its Blockchain's database. A transaction of cryptocurrencies is therefore the transferal of these writing permissions to another user. Blockchains are often developed under an open source license and can therefore be copied and transformed by any group of developers. As anyone can copy and edit the codebase, it is also possible for anyone to edit the rules of the blockchain. Therefore two or more groups can create two implementations with incompatible rulesets. If the different rulesets are constructed to only differ after a certain point in the transaction history, the transactions that happened before that point is compatible with both implementations. Because of this, two blockchains can share the same transaction history up to a certain point, but differ after that point. A ruleset change that creates two incompatible rulesets after a certain point but where the transaction history is shared until the change, is called a Contentious hard fork.

Contentious hard forks occur most often due to technical disagreements regarding the development of the blockchain. As the the transaction history is shared on two blockchains that have split due to a contentious hard fork, a user that possessed cryptocurrency in the shared transaction history before the time of the contentious hard fork, will most likely possess an equivalent amount of cryptocurrency on both blockchains after the contentious hard fork. The combined value of the cryptocurrency holdings after a contentious hard fork can be lower, higher or the same as before the fork. This is primarily due to new structural and technological aspects of each blockchain as well as their communities. 

Forks that occur with respect to cryptocurrencies not acting as selected constituents are treated as any other cryptocurrency. Forks that occur with respect to cryptocurrencies that presently act as constituents of a BlockchainX index and is a selected contentious hard fork creates uncertainty in the pricing of the forked constituent and are therefore treated as follows:

\subsubsection{Selected contentious hard forks}

A selected contentious hard fork is a contentious hard fork where at least two selected exchanges publicly announced it’s support. Contentious hard forks without public announcements from two selected exchanges are treated as any other cryptocurrency.

\begin{enumerate}

\item the price of the forked cryptocurrency is frozen from the time of the fork plus or minus 2 hours, until at least two index exchanges have enabled deposits and withdrawal for both forks.

\item after the criterion of deposits and withdrawal is satisfied, the fork that is eligible to act as a selected constituent is used to determine the index value. In the case where both forks are eligible as selected constituents where both chains have the same validation rules, and its only purpose is chronological ordering of transactions, the fork with the most work done, meaning the most cumulative work (Proof-of-Work) “weight” , is used to determine the index value.

In the case where both forks are eligible as selected constituents where validation rules for transactions differ, the BlockchainX Index Committee will decide the valid chain.

\item the fork that was eligible to act as a selected constituent and used to determine the index value is exempt from criteria 5-8 that concern trade volumes. Furthermore, the fork only needs to fulfill each criterion at the time of rebalancing and not for the standard period of three months.

\item the fork that was not used as a price source can be included in the index as a constituent if it, at the next rebalancing date, fulfills all eligibility criteria except criteria 5-8 that concern trade volumes. Furthermore, the fork only needs to fulfill each criterion at the time of rebalancing and not for the standard period of three months.

\item hard forked eligible index constituents with a fork block number less than two weeks from a rebalancing date will not be rebalanced on the rebalancing date.

\end{enumerate}

A fork that occurs with respect to cryptocurrencies that presently act as constituents of a BlockchainX index can result in none, one or more cryptocurrencies to be included in the indices at the rebalancing date.

\hfill



\subsubsection{Staking}

Writing permissions to a Blockchain’s transaction history are administered by the access to private keys, which are connected to public addresses containing cryptocurrencies. However, only using private keys does not protect the Blockchain from users trying to spend their holdings twice, also known as double-spend attacks. To protect itself from these attacks, all Blockchains implement some form of consensus process that enables the network to reach consensus regarding transaction validity. In order to ensure that users do not corrupt the consensus process, participation must come with a cost that is external to the network. For example, the Bitcoin network consumes electricity, in a process called Proof-of-Work (PoW), in order to protect its transaction history. Staking, or more formally Proof-of-Stake (PoS), is another technology used by Blockchains in order to maintain the immutability of their transaction history. Participants of a Blockchain that implements staking can stake some of their holdings of the network’s cryptocurrency in order to participate in the consensus process. Those who stake and verify transactions in an honest manner are rewarded with new cryptocurrencies, while those who verify transactions that later are deemed invalid are penalized by loosing their stake. Certain Blockchains demand that staked cryptocurrencies are locked for a certain time or that a certain amount of cryptocurrencies are stacked, while others allow the reward of staking to depend on the time the holdings have been staked. The cost of staking is therefore an alternative cost, i.e., the cost incurred due to not being able to invest in other investments with higher return. An index that includes return due to staking forces investors who seek to track the index to stake some of their assets. This creates more complexity and, for that reason, the BlockchainX indices do not include returns due to staking. However, specific purpose indices can be created for clients  where staking revenue is included.

\subsubsection{Airdrops}

Airdrops occur when a Blockchain, or a part of a Blockchain, decides to distribute cryptocurrencies, free of charge, to either their or a different Blockhain’s users. They are most often performed in order to boost network activity or to reward long-term users. Airdrops can come unannounced or they can be disclosed beforehand. Established Blockchains that want to reward old users or boost network activity tend to not announce airdrops, while new networks often announce their airdrops due to marketing reasons. Airdrops often come with a need to perform some tasks in order to obtain the free cryptocurrencies. This can include holding the native asset at a specific date, having to perform a set of transactions on the network, or participating in different surveys. Given the unpredictable nature of airdrops, the BlockchainX indices do not include their return. 


%% Only applies to to the processes used in order to obtain data define as below:
%% ‘contribution of input data’ means providing any input data not readily available to an administrator, or to another person for the purposes of passing to an administrator, that is required in connection with the determination of a benchmark, and is provided for that purpose; 3(8) BMR

%% As I interpret it it is the non-standardized data gathering as well as data that is non-reliable with respect to its availability


%% We need rules that ensure trust in those processes where we use data that is not easily verifiable, examples are
%% free-float computation


\section{Governance and control requirements for supervised contributors}
\subsection{Oversight function}
Vinter Capital has, according to Article 5(1) of the BMR, established a permanent and effective oversight function for all aspects of the provision of benchmarks in the form of an index committee. The members of the oversight function are selected and assured to have, in their entirety, the necessary skills, knowledge and expertise. No member of the committee has been convicted of financial service related offences. The BlockchainX indices are not based on contributors and are thus not subject to contributor-related conflict risks. The oversight function is embedded within the Vinter Capital’s organizational structure to allow it to effectively challenge the management body’s decision. The oversight function has the power to act independently of the administrator, where the Regulation requires it to report to the relevant competent authority any misconduct by contributors or administrators and any anomalous or suspicious input data according to Article 5(3) point (i) of the BMR. The oversight function continuously assures that the administrator can operate exclusively using readily accessible data.
\newline

The oversight function shall operate with integrity and shall have the following responsibilities, which shall be adjusted by Vinter Capital based on the complexity, use and vulnerability of the benchmark:


\begin{enumerate}[label=\alph*)]
    \item reviewing the benchmark's definition and methodology annually which includes, but is not limited to, exchange and constituent criteria, ranking procedures and weighting schemes, data providers and standardized evalutaion procedures.
    \item overseeing any changes to the benchmark methodology and being able to request the administrator to consult on such changes.
    \item overseeing the administrator's control framework, the management and operation of the benchmark. The control framework contains provisions requiring periodic review of the process for contributing input data, effective oversight of the same, and policy on whistleblowing, including appropriate safeguards for whistle-blowers.
    
    As of today, Vinter Capital considers all of its data used for the benchmark as readily available and, therefore, not in need of a code of conduct referred to in Article 15 of the BMR.
    \item reviewing and approving procedures for cessation of the benchmark, including any consultation about a cessation.
    \item overseeing any third party involved in the provision of the benchmark, including calculation or dissemination agents.
    \item assessing internal and external audits or reviews, and monitoring the implementation of identified remedial actions.
    \item if the benchmark at any time becomes based on input data from contributors, monitoring the input data and contributors and the actions of the administrator in challenging or validating contributions of input data.
    \item if the benchmark at any time becomes based on input data from contributors, taking effective measures in respect of any breaches of the code of conduct referred to in Article 15.
    \item reporting to the relevant competent authorities any misconduct by contributors, where the benchmark is based on input data from them, or administrators, of which the oversight function becomes aware, and any anomalous or suspicious input data.
\end{enumerate}

\subsubsection{Constitution of the oversight function}
Vinter Capital will have clear criteria to select members and observers including the evaluation of their expertise and skills (but without publicly disclosing their identity), rules for the meetings of the oversight function and on the participation of staff members therein, the selection of the contact person for the management body and on the interaction with it and arrangements to ensure confidentiality. Vinter Capital will establish procedures to manage the conflicts of interests which may arise due to competing interests of committee members. This list covers the disclosure of conflicts of interest of members of the oversight function, limitations and removal of voting rights from conflicted members as well as the exclusion of members from discussions where they could be conflicted. Furthermore, these procedures forbid members to sit on oversight functions of more than one administrator.

\subsection{Changes}
The procedures for consulting on any proposed material change in Vinter Capital’s methodology as benchmark administrator and the rationale for such changes are included below. This includes a definition of what constitutes a material change and the circumstances in which Vinter Capital is to notify users of any such changes. The procedures required regarding proposed material changes provides for advance notice, with a clear time frame, that gives the opportunity to analyse and comment upon the impact of such proposed material changes. Those comments and Vinter Capital’s response to those comments are made accessible after any consultation, except where confidentiality has been requested by the originator of the comments. 

\subsubsection{Material change}
ESMA allows administrators to define material change and determine the practical aspects of the consultation procedure at their discretion. A material change of a benchmark is any change to the index methodology that would lead to a substantial change in index trajectory. 
\subsubsection{Consultation}
Vinter Capital’s Compliance Department will review any changes to this methodology. The Compliance Department, as well as the independent Oversight Function, has the power to, at any time, request further explanations and information regarding those changes. The Compliance Department will analyse the possible changes with respect to their accuracy, reliability, verifiability, clarity, robustness, transparency, validity and integrity. The Compliance Department will produce a review statement, wherein the compliance of the proposed changes is determined. The statement will be sent to Vinter Capital’s operative department and archived. Material changes must, in addition to being approved by Vinter Capital’s Compliance Department, be approved by the independent Oversight Function in order to be enforced and implemented. 
\subsubsection{Notice}
All material changes are subject to an advance notice published by Vinter Capital. The notice will be sent to users as well as published 60 days prior to the change and will include a clear time frame. Vinter Capital may apply a shorter notice at its own discretion if the affected index is not being used nor is licensed to any third party using it for its financial product(s). All recipients of the notice will be given the opportunity to comment on the proposed change(s). All comments will be published by Vinter Capital except when the commenting party explicitly has requested confidentiality. 

\subsubsection{Discretion}
Vinter Capital has established clear rules identifying how and when discretion
may be exercised in the determination of benchmarks. Vinter Capital may
at its own discretion change input data if it can not be derived from:

\begin{enumerate}[label=\alph*)]
\item
a computational scheme using readily available data, or data contributed under a code of conduct, and that are approved by the oversight function or to be considered of the same standard as those approved by the oversight function or another assessor independent scheme. 
\item a designated assessor or a group of designated assessors whose expertise, excperience as well as characters have been reviewed by the oversight function.  
\end{enumerate}

%% Methodology committee
%% Committee responsible for developing and updating inclusion as well as exclusion criteria for index universe and index computational rules, meets every six month
%% Standardization committe - responsible to assess if inclusion and exclusion criteria can be determined with verifiable data or if contributed data is needed. Are in charge of developed standardized assessment schemes in order to reduce the need for contributed data.
%% Index committee Responsible to assess criteria set forth by the methodology committee. using schemes set forth by the standardization committee. 



% Sections below can be written by Marco and me if he thinks it is necessary, otherwise they can be removed



\section{Compliance statement}
Vinter Capital is compliant with Article 25 and 26 of the BMR and will therefore not publish a compliance statement explaining its reason for non-compliance.


\pagebreak 

\section{Appendix}

\subsection{Index tickers}

%% Flytta tabell till appendix,
%% Each index ticker is constructed as follows: position 1-2 specify index family, position 2-4 specify number of constituents, position 5 specifies weighting scheme and position 7 specifies denomination. 
%% ta bort: These three parameters are reflexted in the ticker, and can be thought of as the X in BlockchainX. 

% ev byta 10 tll X och 5 till V men det är mindre tydligt för ögat

\begin{table}[H]
\caption{Ticker naming convention for the BlockchainX indexes.}
\label{ticker-table}
\begin{tabular}{@{}llll@{}}
\toprule
\textbf{Constituents} & \textbf{Weighting} & \textbf{Denomination} & \textbf{Ticker} \\ \midrule
10                    & Market cap.        & SEK                   & BL10M-S         \\
10                    & Equal              & SEK                   & BL10E-S         \\
5                     & Market cap.        & SEK                   & BL5M-S          \\
5                     & Equal              & SEK                   & BL5E-S          \\
10                    & Market cap.        & EUR                   & BL10M-E         \\
10                    & Equal              & EUR                   & BL10E-E         \\
5                     & Market cap.        & EUR                   & BL5M-E          \\
5                     & Equal              & EUR                   & BL5E-E          \\
10                    & Market cap.        & USD                   & BL10M-U         \\
10                    & Equal              & USD                   & BL10E-U         \\
5                     & Market cap.        & USD                   & BL5M-U          \\
5                     & Equal              & USD                   & BL5E-U          \\
\bottomrule
\end{tabular}
\end{table}

\subsection{Changes to the index methodology}
This table contains all changes to the index methodology after 0180101, when the European Bench- mark Regulation became effective.
\begin{table}[H]
\caption{Changes to the index methodology}
\label{changes-table}
\begin{tabular}{@{}llll@{}}
\toprule
\textbf{Date} & \textbf{Version} & \textbf{Section} & \textbf{Change} \\ \midrule
20181220                   & 0.2      & 3.4 Circulating supply                 & Computation scheme      \\
20181220                     & 0.2              & 4.7.1 Forks                & Criteria for selected contentious hard forks        \\
20181220                     & 0.2       & 4.7.1 Forks                 &  \multicolumn{1}{p{6cm}}{The price of an index constituent is frozen during a 2-hour time window centered at the time point at which the fork will occur.}
\bottomrule
\end{tabular}
\end{table}
\end{document}
